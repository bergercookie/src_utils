% use article styling for this document
\documentclass{article}

% enable system font access
\usepackage{fontspec}
\usepackage{polyglossia}


% styling: Palatino (main text), Helvetica (stress), Code2000 (exotic text) and Asana Math (for math things)
\setmainfont{Palatino Linotype}
\newfontfamily{\maintext}{Palatino Linotype}
\newfontfamily{\stressed}{Helvetica}
\newfontfamily{\exotic}{Arial}
\newfontfamily{\math}{Arial}
\usepackage{fontspec}
\setsansfont{Arial}
\setdefaultlanguage{greek}
% start of actual document
\begin{document}

\section{XeLaTeX documents}
Από ότι συμπαιρένω αυτό θα πρέπει να είναι άμεσο!!
ΝΑ παίζει ελληνικά χωρίς να γράφω κάτι άλλο
XeLaTeX documents, which are processed using the {\stressed xelatex} program, use
document source that is in {\stressed UTF-8} encoding, so that you are free to write
LaTeX
in any script that is supported by {\stressed Unicode}.

\section{multilingual text}

Multilingual text is relatively simple in XeLaTeX (when compared to other versions of
TeX), while still leaving you in total control of the styling (compared to word
processing programs like Microsoft Word or OpenOffice Writer)

\subsection{English}
TeXu
English text is fairly obvious. We pick a main font that supports ascii and we're done!

\subsection{Other languages}

Languages that don't use the``Latin'' script are also not a problem, provided we remember to change
to the correct {\stressed font}. For instance, we can write in Japanese (\exotic
{``Καλημέρα'' τι κάνετε}), or in
Thai ({\exotic ภาษาไทย}), or in Vietnamese ({\exotic tiếng Việt}), or any other number of languages.

\subsection{Automatic font switching}

XeLaTeX even lets you set up automatic font switching, but it's a bit more work because you need
to tell it which fonts to use for which letters, so most people wouldn't consider it basic functionality.

\section{Sectioning}

We can also section our document so that it's easier for the reader to read through. This document
consists of three sections, of which the second section has three subsections.

\end{document}
